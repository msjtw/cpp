\documentclass[13pt]{article}
\usepackage{listings}
\usepackage{xcolor}
\usepackage{hyperref}


\title{CP Algorithms Handbook}
\author{Stanisław Fiedler}
\date{\today}

\definecolor{codegreen}{rgb}{0,0.6,0}
\definecolor{codegray}{rgb}{0.5,0.5,0.5}
\definecolor{codepurple}{rgb}{0.58,0,0.82}
\definecolor{backcolour}{rgb}{0.95,0.95,0.92}

\lstdefinestyle{mystyle}{
    backgroundcolor=\color{backcolour},   
    commentstyle=\color{codegreen},
    keywordstyle=\color{magenta},
    numberstyle=\tiny\color{codegray},
    stringstyle=\color{codepurple},
    basicstyle=\ttfamily\footnotesize,
    breakatwhitespace=false,         
    breaklines=true,                 
    captionpos=b,                    
    keepspaces=true,                 
    numbers=left,                    
    numbersep=5pt,                  
    showspaces=false,                
    showstringspaces=false,
    showtabs=false,                  
    tabsize=4
}

\lstset{style=mystyle}

\begin{document}
\maketitle
\pagebreak

% Optionala 
\tableofcontents
\pagebreak

\section{Sorting and Searching}

\subsection{4SUM}

\section{Dynamic Programming}

\section{Graph Algorithms}

\subsection{Flows and Cuts}
All problems in this section can be solved using the same basic algorithm defined in \ref{sec:MAXFLOW}.

\subsubsection{Maximum Flow}
\label{sec:MAXFLOW}
This implementation of Ford-Fulkerson algorithm (known as Edmonds-Karp algorithm) uses BFS to check if it's possible to expand the flow through the graph and assign depths from the source. Later it uses DFS to expand te flow.
\lstinputlisting[language=C++, firstline=11, lastline=66]{code_snippets/max_flow}

To find the flow:
\lstinputlisting[language=C++, firstline=89, lastline=97]{code_snippets/max_flow}

\subsubsection{Minimum Cut}
To find Minimum Cut in a graph we need to find the Maximum Flow and check which edges connect two created disjont sets of nodes. This modification to finds them:
\lstinputlisting[language=C++, firstline=92, lastline=116]{code_snippets/min_cut}
\subsubsection{Maximum Matching}

\subsection{2SAT}

Given logical formula in the conjunctive normal form:
\[ (a_1 \vee b_1) \wedge  (a_2 \vee b_2) \wedge ... \wedge (a_{n} \vee b_{n})\]
we can eliminate disjunction by repacing each \( (a_i \vee b_i) \) element with pair:
\[ \neg a_i \rightarrow b_i \wedge \neg b_i \rightarrow a_i \]

\section{Range Queries}

\subsection{Segment trees}

BASE size table:
\begin{center}
	\begin{tabular}{ c || c | c | c }
		a          & \(2 \cdot 10^5\) & \(5 \cdot 10^5\) & \(10^6\) \\
		\hline\hline
		\(log_2a\) & 18               & 19               & 20
	\end{tabular}
\end{center}

\subsubsection{Point-Range Trees}
\lstinputlisting[language=C++, firstline=84,lastline=91]{code_snippets/segment_trees}
\lstinputlisting[language=C++, firstline=93,lastline=107]{code_snippets/segment_trees}

\subsubsection{Range-Range Trees}
\begin{enumerate}
	\item range ADD insert, range MAX value query
	      \lstinputlisting[language=C++, firstline=1,lastline=20]{code_snippets/segment_trees}
	      \lstinputlisting[language=C++, firstline=22,lastline=39]{code_snippets/segment_trees}
	\item range ADD insert, range SUM qyerry
	      \lstinputlisting[language=C++, firstline=41,lastline=63]{code_snippets/segment_trees}
	      \lstinputlisting[language=C++, firstline=65,lastline=83]{code_snippets/segment_trees}
\end{enumerate}
\section{Tree Algorithms}
\section{String Algorithms}
\section{Mathematics}
\section{Geometry}

% section Geometry (end)
\section{Other Algorithms}


%--/Paper--

\end{document}
